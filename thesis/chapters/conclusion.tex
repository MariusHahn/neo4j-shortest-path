\chapter{Conclusion}

The research presented in the thesis at hand demonstrates the successful integration of Customizable Contraction Hierarchies (CCH) into the Neo4j graph database, aimed at enhancing the efficiency of shortest path queries.
The adoption of CCH into Neo4j led to significant improvements in query processing speeds, particularly in road networks where certain vertices play a more crucial role in the multitude of shortest paths.
The study also explores  the impact of updating edge weights on the performance of the CCH index.
It was found that despite multiple updates, the CCH index maintained its efficiency, proving its robustness in dynamic environments where data changes frequently.
This aspect is particularly important in real-world applications where data is not static but continuously evolving.
The results showed that the index graph, could be efficiently managed, thus ensuring quick access and processing.
An important aspect of this research was to keep the integration of CCH into Neo4j loosely coupled, thereby allowing for easier portability to other graph databases in the future.
This approach not only enhances the adaptability of the research findings but also paves the way for future developments in this field.
In conclusion, this thesis makes a contribution to the field of graph databases by improving the efficiency and performance of shortest path queries in Neo4j through the integration of CCH.
It opens up new avenues for further research, particularly in the area of dynamic data handling and cross-platform applicability of such advanced querying techniques.

