\chapter{Related Work} 

As this is manly a database paper, we want to divide this chapter in two main sections \nameref{sec:algorithmic_history} and \nameref{sec:related_work:database}. \nameref{sec:algorithmic_history} 
will give some basic overview what has been published regarding index structures to speed up shortest path queries for graphs. \nameref{sec:related_work:database} will try 
to give an overview of efforts that have been made to make \cite[Customizable Contraction Hierarchies]{CCH} it suitable for graph databases.

\section[Algorithmic History]{Algorithmic History} \label{sec:algorithmic_history}

\cite[Contraction Hierarchies]{Geisberger_2012} or CH is heavly influenced by the idea of the \cite[Transit-Node]{Bast_2007} approach and as transit node approach itself is a
technique to speed up the \cite[Dijkstra Algorithm]{Dijkstra_1959} which is the most basic and robust algorithm to find shortest path in graphs. CH goes back to the diploma thesis of \cite[Geisberger]{Geisberger} in 2008. The \cite[Transit-Node]{Bast_2007} approach
tries to find vertex inside the graph that are more important than others. Important in this case means, these are vertices that lie on many shortest paths. This speeds up 
especially long distance queries, as one only needs to calculate the distance to the the next transit node of the source and target vertex because the shortest paths between 
the transit or access nodes will be known. \\ 
CH goes even further on the idea of having important vertices. It applies an importance to each vertex in the graph a so called rank. Furthermore it add edges to the graph
, so called shortcut, that preserve the shortest path property of the graph in case a vertex that is contracted lies on a shortest path between other. When querying a shortest
path CH uses a modified bidirectional-dijkstra that is restricted to only visit nodes that are of higher importance, or rank, than the its about to expand next.
This method is able to retrieve shortest paths of vertices that have a high spacial distance, however, it is rather static. In case a new edge is added or an edge weight is updated, 
it might be necessary to recontract the whole graph to preserve the shortest path property. \\
In 2016 \cite[Customization Contraction Hierarchies]{CCH} or CCH was published. The approach is the same, but in CCH shortcuts are not only added if the contraction violtes the shortest,
they are added if there had been a connection between its neighbors through the just contracted vertex and these neighbors do not own a direct connection through an already existing edge.
The shortcut weights are later on calculated through the lowers triangle. Additionally the \cite[Customization Contraction Hierarchies]{CCH} provides an update approach that only updates,
edges that are affected by a weight change.

\section{Contraction Hierarchies Database History}\label{sec:related_work:database}

There is one bachelor thesis by Nicolai D'Effremo \cite[Some text]{DEffremo2019} that has implemented a version on \cite[Contraction Hierarchies]{Geisberger_2012} for Neo4j, one 
of the most used graph databases of today in 2023. This implementation shows that even in CH an index structure is also worth pursuing in a database context, as the speedup 
of shortest path queries paired with a reasonable preprocessing time. \cite{Zickenberg2021} wrote a showed in his bachelor thesis that it is even possible to restricted these
queries with label constraints. Although CH and CCH have little difference, sadly we could not use much of the code provided as it
was deeply integrated into the Neo4j-Platform and since that there have been two major release updates with breaking changes which make it nearly impossible to reuse any of
this code.\\

Finally there is \cite[Mobile Route Planning]{Sanders} by Peter Sanders, Dominik Schultes, and Christian Vetter. In this paper it is described how one can efficiently store
the a CH index structure on a hard drive. It states an interesting technique to how store edge that are likely to be read sequentially spatially close on the hard drive which 
makes read operations that have to be done during query time fast. The motivation of \cite[Mobile Route Planning]{Sanders} through was slightly different. They came up with this
idea because computation power on mobile devices is limited, so they could precalculate the CH index on a server and then later distribute it to a mobile device.
