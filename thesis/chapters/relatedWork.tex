\chapter{Related Work} 

As this is mainly a database paper, we want to divide this chapter in two main sections \nameref{sec:algorithmic_history} and \nameref{sec:related_work:database}. \nameref{sec:algorithmic_history} 
that will give some basic overview what has been published regarding index structures to speed up shortest path queries for graphs. \nameref{sec:related_work:database} we will try 
to give an overview of efforts that have been made to make \cite[Customizable Contraction Hierarchies]{CCH} it suitable for graph databases.

\section[Algorithmic History]{Algorithmic History} \label{sec:algorithmic_history}

\cite[Contraction Hierarchies]{Geisberger_2012} or CH is heavily influenced by the idea of the \cite[Transit-Node]{Bast_2007} approach and as transit node approach itself is a
technique to speed up \cite[Dijkstras Algorithm]{Dijkstra_1959}, which is the most basic and robust algorithm to find shortest path in graphs. CH goes back to the diploma thesis of \cite[Geisberger]{Geisberger} in 2008. The \cite[Transit-Node]{Bast_2007} approach
tries to find vertices inside the graph that are more important than others. Important in this case means, these are vertices that reside on many shortest paths. This speeds up 
especially long distance queries, as one only needs to calculate the distance to the the next transit node of the source and target vertex as the shortest paths between 
the transit or access nodes will be known. \\ 
CH goes even further on the idea of having important vertices. It applies an importance to each vertex in the graph a so called rank. Furthermore it adds edges to the graph,
so called shortcuts, that preserve the shortest path property of the graph in case a vertex that is contracted resides on a shortest path between others. When querying a shortest
path CH uses a modified bidirectional-dijkstra that is restricted to only visit nodes that are of higher importance, or rank, than the its about to expand next.
This method is able to retrieve shortest paths of vertices that have a high spacial distance, however, it is rather static. In case a new edge is added or an edge weight is updated, 
it might be necessary to recontract the whole graph to preserve the shortest path property. \\
In 2016 \cite[Customization Contraction Hierarchies]{CCH} or CCH was published. The approach is the same, but in CCH shortcuts are not only added if the contraction violates the shortest path property,
they are added if there had been a connection between its neighbors through the just contracted vertex and these neighbors do not own a direct connection through an already existing edge.
The shortcut weights are later on calculated through the lowers triangle. Additionally the \cite[Customization Contraction Hierarchies]{CCH} provides an update approach that only updates,
edges that are affected by a weight change.

\section{Contraction Hierarchies Database History}\label{sec:related_work:database}

There is one bachelor thesis by Nicolai D'Effremo \cite[Some text]{DEffremo2019} that has implemented a version on \cite[Contraction Hierarchies]{Geisberger_2012} for Neo4j, one 
of the most used graph databases of today in 2023. This implementation shows that even in for databases CH is an index structure worth pursuing, as there was a tremendous speedup 
of shortest path queries paired with a reasonable preprocessing time. \cite{Zickenberg2021} showed in his bachelor thesis that it is even possible to restricted these
queries with label constraints. Although CH and CCH have little difference, sadly we could not use much of the code provided by there works. It
was deeply integrated into the Neo4j-Platform and since then two major release updates happened that have breaking changes which make it nearly impossible to reuse any of
this code.\\

Finally there is \cite[Mobile Route Planning]{Sanders} by Peter Sanders, Dominik Schultes, and Christian Vetter. In this paper it is described how one can efficiently store
the a CH index structure on a hard drive. It states an interesting technique to how store edge that are likely to be read sequentially spatially close on the hard drive which 
makes read operations that have to be done during query time fast. The motivation of \cite[Mobile Route Planning]{Sanders} through was slightly different. They came up with this
idea because computation power on mobile devices is limited, so they could precalculate the CH index on a server and then later distribute it to a mobile device.
\\
We will use parts of this idea and partly port it to our database context as we suppose there are many similarities.
